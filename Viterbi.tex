\subsection{Viterbi decoder}

Another simulation workflow evaluated the performance of the Viterbi decoder in identifying
the sequence of true states when only observation sequences are known.  The Viterbi decoder is a
 dynamic programming based algorithm which combines the observation probability densities / distributions
with the state transition probabilities 
to determine the most likely state sequencce.
 

To measure the accuracy of state decoding, the sequences of states (both true and decoded) were encoded 
as strings of one character per state and the string Levanstein string edit distance was computed.
The string edit distance was divided by the number of characters (states) in each sequence.

The Viterbi algorithm was evaluated using a set of HMMs which were perturned by different amounts from the 
initial HMM which generated the sequences.  For a pertubation of 0, the Viterbi algorithm always returned 
the same estimated sequence for each simluated observation sequence.  
For perturbations greater than 0, the direction of the perturbation was randomized as described above so 
15 repetations were performed to get an average performance in terms length-adjusted SED.
As with baum wehlch system identification results above, 
an additional set of computations was performed for randomly initialized HMM transiton parameters.

Observations were generated with a set of 5 Ratios:
\[
R = \{0.0, 0.25, 1.0, 2.5, 5.0\}
\]
Ratio, $R$, encodes the ease of decoding the states from observation values alone without state-transition
probabilities. 

Figure \ref{res_SED_v_Ratio_16state.png} shows average SED for the 5 values of observation Ratio, $R$ in the 15-state model.
All HMM perterbation values are included in this dataset.  Perhaps surprisingly there seems to be no effect of the obervation Ratio on 
mean SED. 



Figure \ref{res_SED_v_Pert_16state_all.png} shows average SED for 4 values of HMM perturbation,  plus the random HMM $A$ matrix for 
all Ratios in the 16-state model.  This time there is a notable trend of worse state tracking results for higher model perturbations 
and especially for the random HMM state-transition matrix. 







